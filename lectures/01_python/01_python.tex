% !TEX options=--shell-escape

\documentclass[10pt]{beamer}

\usepackage[T1]{fontenc}
\usepackage{beramono}

\usepackage{anyfontsize}
\usepackage{booktabs}
\usepackage{etoolbox}
\usepackage{hyperref}
\usepackage{setspace}
\usepackage{tcolorbox}
\usepackage{texlogos}
\usepackage{xcolor}
\usepackage{xspace}

\usetheme{metropolis}
\usepackage{appendixnumberbeamer}

\newcommand\cpp[1]{\cpluspluslogo #1}
\newcommand\mono[1]{\texttt{\footnotesize{\detokenize{#1}}}\xspace}
\newcommand\pybind{\mono{pybind11}}
\newcommand\codeskip{\\\vspace*{0.5em}}
\newcommand\darrow{\parbox{3cm}{\centering$\downarrow$}\smallskip}
\newcommand\supertiny{\fontsize{3}{3}\selectfont}
\newcommand\smaller{\fontsize{6}{6}\selectfont}

\definecolor{urlcolor}{rgb}{0.75,0.45,0.15}
\renewcommand\UrlFont{\footnotesize\ttfamily\color{urlcolor}}

\definecolor{lgreen}{rgb}{0.8,0.9,0.8}
\definecolor{lred}{rgb}{0.9,0.8,0.8}
\definecolor{lgrey}{rgb}{0.6,0.6,0.6}
\definecolor{lblue}{rgb}{0.8,0.8,0.9}

\definecolor{dgreen}{rgb}{0.2,0.6,0.2}
\definecolor{dred}{rgb}{0.6,0.2,0.2}

\newcommand\hlbox[2]{\setlength{\fboxsep}{0pt}\colorbox{#1}{#2}}
\newcommand\hlcode[1]{\hlbox{lgreen}{\detokenize{#1}}}
\newcommand\hlcodealt[1]{\hlbox{lblue}{\detokenize{#1}}}
\newcommand\dimcode[1]{\hlbox{lgrey}{\detokenize{#1}}}

\usepackage{minted}
\usemintedstyle{trac}
\newminted{cpp}{autogobble,escapeinside=@@,fontsize=\footnotesize}
\newminted{python}{autogobble,escapeinside=@@,fontsize=\footnotesize}
\newminted{sh}{autogobble,escapeinside=@@,fontsize=\footnotesize}
\BeforeBeginEnvironment{minted}{
    \begin{tcolorbox}[colback=black!3!white,colframe=black!16!white,boxrule=0.3mm]
    }
\AfterEndEnvironment{minted}{\end{tcolorbox}}
\AtBeginEnvironment{minted}{\setstretch{1.2}}

\title{Python Tutorial}
\subtitle{RGL - CS328 Numerical Methods for Visual Computing}
\date{\today}
\author{Tizian Zeltner\\
Jan Bednarik}
%\institute{Susquehanna International Group\\
%[6\baselineskip]{euro\textbf{python} 2017}}

\begin{document}

\maketitle

%%%%%%%%%%%%%%%%%%%%%%%%%%%%%%%%%%%%%%%%%%%%%%%%%%%%%%%%%%%%%%%%%%%%%%%%%%%%%
%%%%%%%%%%%%%%%%%%%%%%%%%%%%%%%%%%%%%%%%%%%%%%%%%%%%%%%%%%%%%%%%%%%%%%%%%%%%%
%%%%%%%%%%%%%%%%%%%%%%%%%%%%%%%%%%%%%%%%%%%%%%%%%%%%%%%%%%%%%%%%%%%%%%%%%%%%%

\section{Introduction}

%%%%%%%%%%%%%%%%%%%%%%%%%%%%%%%%%%%%%%%%%%%%%%%%%%%%%%%%%%%%%%%%%%%%%%%%%%%%%

\begin{frame}{Extension modules}
    CPython \textit{extension module}: Python module not written in Python.

    \pause

    Most often written in C or \cpp{}.

    \pause

    \hrulefill

    \textit{Why bother?}

    \begin{itemize}
        \item \pause Interfacing with existing libraries
        \item \pause Writing performance-critical code
        \item \pause Mirroring library API in Python to aid prototyping
        \item \pause Running tests for non-Python libraries in Python
    \end{itemize}
\end{frame}

%%%%%%%%%%%%%%%%%%%%%%%%%%%%%%%%%%%%%%%%%%%%%%%%%%%%%%%%%%%%%%%%%%%%%%%%%%%%%

\begin{frame}{Python C API}
    It is possible to write CPython extension modules in pure C, but...

    \begin{itemize}
        \item \pause Manual refcounting
        \item \pause Manual exception handling
        \item \pause Boilerplate to define functions and modules
        \item \pause High entry barrier, prone to programmer errors
        \item \pause Differences in the API between Python versions
    \end{itemize}
\end{frame}

%%%%%%%%%%%%%%%%%%%%%%%%%%%%%%%%%%%%%%%%%%%%%%%%%%%%%%%%%%%%%%%%%%%%%%%%%%%%%

\begin{frame}{Cython}
    Cython: "let's write C extensions as if it was Python".

    \pause

    Why not?

    \begin{itemize}
        \item \pause It's neither C nor Python
        \item \pause A 2-line Cython module can be transpiled into 2K lines of C
        \item \pause Two build steps (\mono{.pyx} $\rightarrow$ \mono{.c}, \mono{.c} $\rightarrow$ \mono{.so}); poor IDE support
        \item \pause Limited \cpp{} support (scoped enums, non-type
            template parameters, templated overloads, variadic templates,
            universal references, etc).
        \item \pause Limited support for generic code beyond fused types
        \item \pause Have to create stubs for anything outside standard library
        \item \pause Great for wrapping a few functions, not so great for large codebases
        \item \pause Debugging compiled Cython extensions is \textbf{pain}
    \end{itemize}
\end{frame}

%%%%%%%%%%%%%%%%%%%%%%%%%%%%%%%%%%%%%%%%%%%%%%%%%%%%%%%%%%%%%%%%%%%%%%%%%%%%%

\begin{frame}[fragile]{Cython}

    \begin{pythoncode}
        def f(n: int):
            for i in range(n):  # <-------
                pass
    \end{pythoncode}

    \pause \darrow

    \begin{shcode}
        $ cythonize example.pyx
    \end{shcode}

    \pause \darrow

    \begin{shcode}
      /\___/\
     ( o   o )
     (  =^=  )
     (        )
     (         )
     (          )))))))))))
    \end{shcode}

\end{frame}

%%%%%%%%%%%%%%%%%%%%%%%%%%%%%%%%%%%%%%%%%%%%%%%%%%%%%%%%%%%%%%%%%%%%%%%%%%%%%

\begin{frame}[fragile]{Cython}

\begin{minted}[fontsize=\supertiny]{cpp}
/* "example.pyx":2
 * def f(n: int):
 *     for i in range(n):             # <<<<<<<<<<<<<<
 *         pass
 *
 */
  __pyx_t_1 = PyTuple_New(1); if (unlikely(!__pyx_t_1)) __PYX_ERR(0, 2, __pyx_L1_error)
  __Pyx_GOTREF(__pyx_t_1);
  __Pyx_INCREF(__pyx_v_n);
  __Pyx_GIVEREF(__pyx_v_n);
  PyTuple_SET_ITEM(__pyx_t_1, 0, __pyx_v_n);
  __pyx_t_2 = __Pyx_PyObject_Call(__pyx_builtin_range, __pyx_t_1, NULL); if (unlikely(!__pyx_t_2)) __PYX_ERR(0, 2, __pyx_L1_error)
  __Pyx_GOTREF(__pyx_t_2);
  __Pyx_DECREF(__pyx_t_1); __pyx_t_1 = 0;
  if (likely(PyList_CheckExact(__pyx_t_2)) || PyTuple_CheckExact(__pyx_t_2)) {
    __pyx_t_1 = __pyx_t_2; __Pyx_INCREF(__pyx_t_1); __pyx_t_3 = 0;
    __pyx_t_4 = NULL;
  } else {
    __pyx_t_3 = -1; __pyx_t_1 = PyObject_GetIter(__pyx_t_2); if (unlikely(!__pyx_t_1)) __PYX_ERR(0, 2, __pyx_L1_error)
    __Pyx_GOTREF(__pyx_t_1);
    __pyx_t_4 = Py_TYPE(__pyx_t_1)->tp_iternext; if (unlikely(!__pyx_t_4)) __PYX_ERR(0, 2, __pyx_L1_error)
  }
  __Pyx_DECREF(__pyx_t_2); __pyx_t_2 = 0;
  for (;;) {
    if (likely(!__pyx_t_4)) {
      if (likely(PyList_CheckExact(__pyx_t_1))) {
        if (__pyx_t_3 >= PyList_GET_SIZE(__pyx_t_1)) break;
        #if CYTHON_ASSUME_SAFE_MACROS && !CYTHON_AVOID_BORROWED_REFS
        __pyx_t_2 = PyList_GET_ITEM(__pyx_t_1, __pyx_t_3); __Pyx_INCREF(__pyx_t_2); __pyx_t_3++; if (unlikely(0 < 0)) __PYX_ERR(0, 2, __pyx_L1_error)
        #else
        __pyx_t_2 = PySequence_ITEM(__pyx_t_1, __pyx_t_3); __pyx_t_3++; if (unlikely(!__pyx_t_2)) __PYX_ERR(0, 2, __pyx_L1_error)
        __Pyx_GOTREF(__pyx_t_2);
        #endif
      } else {
        if (__pyx_t_3 >= PyTuple_GET_SIZE(__pyx_t_1)) break;
        #if CYTHON_ASSUME_SAFE_MACROS && !CYTHON_AVOID_BORROWED_REFS
        __pyx_t_2 = PyTuple_GET_ITEM(__pyx_t_1, __pyx_t_3); __Pyx_INCREF(__pyx_t_2); __pyx_t_3++; if (unlikely(0 < 0)) __PYX_ERR(0, 2, __pyx_L1_error)
        #else
        __pyx_t_2 = PySequence_ITEM(__pyx_t_1, __pyx_t_3); __pyx_t_3++; if (unlikely(!__pyx_t_2)) __PYX_ERR(0, 2, __pyx_L1_error)
        __Pyx_GOTREF(__pyx_t_2);
        #endif
      }
    } else {
      __pyx_t_2 = __pyx_t_4(__pyx_t_1);
      if (unlikely(!__pyx_t_2)) {
        PyObject* exc_type = PyErr_Occurred();
        if (exc_type) {
          if (likely(exc_type == PyExc_StopIteration || PyErr_GivenExceptionMatches(exc_type, PyExc_StopIteration))) PyErr_Clear();
          else __PYX_ERR(0, 2, __pyx_L1_error)
        }
        break;
      }
      __Pyx_GOTREF(__pyx_t_2);
    }
    __Pyx_XDECREF_SET(__pyx_v_i, __pyx_t_2);
    __pyx_t_2 = 0;
  }
  __Pyx_DECREF(__pyx_t_1); __pyx_t_1 = 0;
\end{minted}

\end{frame}

%%%%%%%%%%%%%%%%%%%%%%%%%%%%%%%%%%%%%%%%%%%%%%%%%%%%%%%%%%%%%%%%%%%%%%%%%%%%%

\begin{frame}[fragile]{Cython}
    \begin{pythoncode}
        def f(n: int):
            @\hlcode{cdef int i}@
            for i in range(n):
                pass
    \end{pythoncode}

    \pause \darrow

\begin{cppcode}
  __pyx_t_1 = __Pyx_PyInt_As_long(__pyx_v_n);
  if (unlikely((__pyx_t_1 == (long)-1) && PyErr_Occurred()))
    __PYX_ERR(0, 3, __pyx_L1_error)
  for (__pyx_t_2 = 0; __pyx_t_2 < __pyx_t_1; __pyx_t_2+=1) {
    __pyx_v_i = __pyx_t_2;
  }
\end{cppcode}

\end{frame}

%%%%%%%%%%%%%%%%%%%%%%%%%%%%%%%%%%%%%%%%%%%%%%%%%%%%%%%%%%%%%%%%%%%%%%%%%%%%%

\begin{frame}{Boost.Python}

    Why not?

    \begin{itemize}
        \item \pause Requires building \mono{boost_python} library
        \item \pause \ldots (!) which requires Boost (full library is 1.5M LOC headers)
        \item \pause \ldots and uses SCons for building (Python2-only build tool)
        \item \pause Relies heavily on Boost.MPL due to being stuck in \cpp{03}
        \item \pause \ldots so large extension modules may take very long to compile
        \item \pause \ldots and resulting binaries may end up being very big
        \item \pause $<200$ commits in the last 5 years
    \end{itemize}

    \pause

    (still a great library if you're already
    using Boost; \pybind's syntax and initial API design were heavily
    influenced by Boost.Python)

\end{frame}

%%%%%%%%%%%%%%%%%%%%%%%%%%%%%%%%%%%%%%%%%%%%%%%%%%%%%%%%%%%%%%%%%%%%%%%%%%%%%

\begin{frame}{pybind11}

    \pybind is a lightweight header-only library that allows
    interacting with Python interpreter and writing Python
    extension modules in modern \cpp{}.

    \begin{itemize}
        \item \pause Header-only; no dependencies; doesn't require specific build tools
        \item \pause 5K LOC core codebase (8K entire library)
        \item \pause Heavily optimized for binary size; fast compile time
        \item \pause GCC, Clang, MSVS or ICC (Linux, macOS, Windows)
        \item \pause CPython 2.7, CPython 3.x, \alert<7>{PyPy} \pause
        \item \pause Support for \cpp{11}, \cpp{14} and \cpp{17} language features
        \item \pause Support for NumPy without having to include NumPy headers
        \item \pause Support for embedding Python interpreter
        \item \pause \small{
                STL data types, overloaded functions, enumerations,
                callbacks, iterators and ranges, single and multiple inheritance,
                smart pointers, custom operators, automatic refcounting, capturing
                lambdas, function vectorization, arbitrary exception types, virtual
                class wrapping, etc \ldots
            }
    \end{itemize}

    Link: \url{http://github.com/pybind/pybind11}

\end{frame}

%%%%%%%%%%%%%%%%%%%%%%%%%%%%%%%%%%%%%%%%%%%%%%%%%%%%%%%%%%%%%%%%%%%%%%%%%%%%%
%%%%%%%%%%%%%%%%%%%%%%%%%%%%%%%%%%%%%%%%%%%%%%%%%%%%%%%%%%%%%%%%%%%%%%%%%%%%%
%%%%%%%%%%%%%%%%%%%%%%%%%%%%%%%%%%%%%%%%%%%%%%%%%%%%%%%%%%%%%%%%%%%%%%%%%%%%%

\section{Hello, World!}

%%%%%%%%%%%%%%%%%%%%%%%%%%%%%%%%%%%%%%%%%%%%%%%%%%%%%%%%%%%%%%%%%%%%%%%%%%%%%

\begin{frame}[fragile]{First things first}

    Requirements:

    \begin{itemize}
        \item CPython 2.7.x, 3.x or PyPy $\geqslant 5.7$, with headers
        \item \pybind package installed (\mono{pip install pybind11})
        \item Non-ancient compiler (Clang $\geqslant 3.3$, GCC $\geqslant 4.8$, MSVS $\geqslant 2015$)
    \end{itemize}

    \pause

    Boilerplate (will be omitted in most further examples):

    \begin{cppcode}
        #include <pybind11/pybind11.h>

        namespace py = pybind11;

        PYBIND11_MODULE(example, m) {
            ...
        }
    \end{cppcode}

\end{frame}

%%%%%%%%%%%%%%%%%%%%%%%%%%%%%%%%%%%%%%%%%%%%%%%%%%%%%%%%%%%%%%%%%%%%%%%%%%%%%

\begin{frame}[fragile]{Let's write a module}

    Let's bind a \mono{C} function that adds two integers:

    \begin{cppcode}
        int add(int a, int b) {
            return a + b;
        }

        PYBIND11_MODULE(myadd, m) {
            m.def("add", &add, "Add two integers.");
        }
    \end{cppcode}

    \pause

    \ldots or, \cpp{11} style:

    \begin{cppcode}
        PYBIND11_MODULE(myadd, m) {
            m.def("add", [](int a, int b) { return a + b; },
                  "Add two integers.");
        }
    \end{cppcode}

\end{frame}

%%%%%%%%%%%%%%%%%%%%%%%%%%%%%%%%%%%%%%%%%%%%%%%%%%%%%%%%%%%%%%%%%%%%%%%%%%%%%

\begin{frame}[fragile]{Trying it out}

    After the code is compiled, it can be used like a normal Python module:

    \begin{pythoncode}
        >>> from myadd import add

        >>> help(add)
        add(arg0: int, arg1: int) -> int

        Add two integers.

        >>> add(1, 2)
        3

        >>> add('foo', 'bar')
        TypeError: add(): incompatible function arguments.
        The following argument types are supported:
            1. (arg0: int, arg1: int) -> int

        Invoked with: 'foo', 'bar'
    \end{pythoncode}

\end{frame}

%%%%%%%%%%%%%%%%%%%%%%%%%%%%%%%%%%%%%%%%%%%%%%%%%%%%%%%%%%%%%%%%%%%%%%%%%%%%%

\begin{frame}{Compiling a module}

    There's a few possible ways to build a \pybind module...

\end{frame}

%%%%%%%%%%%%%%%%%%%%%%%%%%%%%%%%%%%%%%%%%%%%%%%%%%%%%%%%%%%%%%%%%%%%%%%%%%%%%

\begin{frame}[fragile]{Compiling a module -- manually}

    \textbf{Linux} (Python 3):

    \begin{shcode}
        $ c++ -O3 -shared -std=c++11 -fPIC
            $(python -m pybind11 --includes) myadd.cpp
            -o myadd$(python3-config --extension-suffix)
    \end{shcode}

    If the build succeeds, it will create a binary module like this:

    \begin{shcode}
        myadd.cpython-36m-x86_64-linux-gnu.so
    \end{shcode}

    \textbf{macOS}: same as above, plus \mono{-undefined dynamic_lookup} flag.

    \textbf{Windows}: possible but not fun.

\end{frame}

%%%%%%%%%%%%%%%%%%%%%%%%%%%%%%%%%%%%%%%%%%%%%%%%%%%%%%%%%%%%%%%%%%%%%%%%%%%%%

\begin{frame}[fragile]{Compiling a module -- distutils}

    Integrating into \mono{setup.py}:

    \begin{pythoncode}
        from setuptools import setup, Extension
        from setuptools.command import build_ext
        from pybind11 import get_include

        setup(
            ...,
            ext_modules=[
                Extension(
                    'myadd', ['myadd.cpp'],
                    include_dirs=[get_include()],
                    language='c++',
                    extra_compile_args=['-std=c++11']
                )
            ],
            cmdclass={'build_ext': build_ext.build_ext}
        )
    \end{pythoncode}

\end{frame}

%%%%%%%%%%%%%%%%%%%%%%%%%%%%%%%%%%%%%%%%%%%%%%%%%%%%%%%%%%%%%%%%%%%%%%%%%%%%%

\begin{frame}[fragile]{Compiling a module -- ipybind}

    % TODO: add ipybind url
    In IPython console or Jupyter notebook (requires installing \mono{ipybind}):

    \begin{pythoncode}
        %load_ext ipybind
    \end{pythoncode}

    \begin{pythoncode}
        %%pybind11

        PYBIND11_MODULE(myadd, m) {
            m.def("add", [](int a, int b) { return a + b; },
                  "Add two integers.");
        }
    \end{pythoncode}

    After the module is built, its contents are imported automatically:

    \begin{pythoncode}
        >>> add(1, 2)
        3
    \end{pythoncode}

\end{frame}

%%%%%%%%%%%%%%%%%%%%%%%%%%%%%%%%%%%%%%%%%%%%%%%%%%%%%%%%%%%%%%%%%%%%%%%%%%%%%

\begin{frame}[fragile]{Compiling a module -- CMake}

    In a CMake project:

    \begin{shcode}
        pybind11_add_module(myadd myadd.cpp)
    \end{shcode}

\end{frame}

%%%%%%%%%%%%%%%%%%%%%%%%%%%%%%%%%%%%%%%%%%%%%%%%%%%%%%%%%%%%%%%%%%%%%%%%%%%%%
%%%%%%%%%%%%%%%%%%%%%%%%%%%%%%%%%%%%%%%%%%%%%%%%%%%%%%%%%%%%%%%%%%%%%%%%%%%%%
%%%%%%%%%%%%%%%%%%%%%%%%%%%%%%%%%%%%%%%%%%%%%%%%%%%%%%%%%%%%%%%%%%%%%%%%%%%%%

\section{Simple classes}

%%%%%%%%%%%%%%%%%%%%%%%%%%%%%%%%%%%%%%%%%%%%%%%%%%%%%%%%%%%%%%%%%%%%%%%%%%%%%

\begin{frame}[fragile]{\cpp{} example class}

    Let's create Python bindings for a simple HTTP response class:

    \begin{cppcode}
        #include <string>

        struct Response {
            int status;
            std::string reason;
            std::string text;

            Response(int status,
                     std::string reason,
                     std::string text = "")
                : status(status)
                , reason(std::move(reason))
                , text(std::move(text))
            {}

            Response() : Response(200, "OK")  {}
        };
    \end{cppcode}

\end{frame}

%%%%%%%%%%%%%%%%%%%%%%%%%%%%%%%%%%%%%%%%%%%%%%%%%%%%%%%%%%%%%%%%%%%%%%%%%%%%%

\begin{frame}[fragile]{Binding the type}

    \begin{cppcode}
        struct Response {
            ...
        }
    \end{cppcode}

    \pause \darrow

    \begin{cppcode}
        PYBIND11_MODULE(response, m) {
            @\hlcode{py::class_}@<Response>(m, "Response");
        }
    \end{cppcode}

\end{frame}

%%%%%%%%%%%%%%%%%%%%%%%%%%%%%%%%%%%%%%%%%%%%%%%%%%%%%%%%%%%%%%%%%%%%%%%%%%%%%

\begin{frame}[fragile]{Constructors}

    \begin{cppcode}
        struct Response {
            ...
            Response(int status,
                     std::string reason,
                     std::string text = "");
            Response();
        };
    \end{cppcode}

    \pause \darrow

    \begin{cppcode}
        py::class_<Response>(m, "Response")
            @\hlcode{.def}@(py::init<>())
            @\hlcode{.def}@(py::init<int, std::string>())
            @\hlcode{.def}@(py::init<int, std::string, std::string>());
    \end{cppcode}

\end{frame}

%%%%%%%%%%%%%%%%%%%%%%%%%%%%%%%%%%%%%%%%%%%%%%%%%%%%%%%%%%%%%%%%%%%%%%%%%%%%%

\begin{frame}[fragile]{Instance attributes}

    \begin{cppcode}
        struct Response {
            ...
            int status;
            std::string reason;
            std::string text;
        };
    \end{cppcode}

    \pause \darrow

    \begin{cppcode}
        py::class_<Response>(m, "Response")
            ...
            @\hlcode{.def_readonly}@("status", &Response::status)
            @\hlcode{.def_readonly}@("reason", &Response::reason)
            @\hlcode{.def_readonly}@("text", &Response::text);
    \end{cppcode}

\end{frame}

%%%%%%%%%%%%%%%%%%%%%%%%%%%%%%%%%%%%%%%%%%%%%%%%%%%%%%%%%%%%%%%%%%%%%%%%%%%%%

\begin{frame}[fragile]{Properties}

    \begin{cppcode}
        struct Response {
            ...
            bool ok() const {
                return status >= 200 && status < 400;
            }
        };
    \end{cppcode}

    \pause \darrow

    \begin{cppcode}
        py::class_<Response>(m, "Response")
            ...
            @\hlcode{.def_property_readonly}@("ok", &Response::ok);
    \end{cppcode}

\end{frame}

%%%%%%%%%%%%%%%%%%%%%%%%%%%%%%%%%%%%%%%%%%%%%%%%%%%%%%%%%%%%%%%%%%%%%%%%%%%%%

\begin{frame}[fragile]{Operators}

    \begin{cppcode}
        bool operator==(const Response& r1,
                        const Response& r2) {
            return r1.status == r2.status
                && r1.reason == r2.reason
                && r1.text == r2.text;
        }
    \end{cppcode}

    \pause \darrow

    \begin{cppcode}
        py::class_<Response>(m, "Response")
            ...
            @\hlcode{.def}@("__eq__", [](const Response& self,
                              const Response& other) {
                return self == other;
            });
    \end{cppcode}

\end{frame}

%%%%%%%%%%%%%%%%%%%%%%%%%%%%%%%%%%%%%%%%%%%%%%%%%%%%%%%%%%%%%%%%%%%%%%%%%%%%%

\begin{frame}[fragile]{Operators (\mono{py::self})}

    Wrapping operators is a very common thing to do:

    \begin{cppcode}
        ...

        .def("__eq__", [](const Response& self,
                          const Response& other) {
            return self == other;
        }, py::is_operator())
    \end{cppcode}

    \pause

    \ldots so there's a shortcut:

    \begin{cppcode}
        #include <pybind11/operators.h>

        ...

        .def(py::self == py::self)
    \end{cppcode}

    (also works with arithmetic operators, binary operators, etc.)

\end{frame}

%%%%%%%%%%%%%%%%%%%%%%%%%%%%%%%%%%%%%%%%%%%%%%%%%%%%%%%%%%%%%%%%%%%%%%%%%%%%%

\begin{frame}[fragile]{Methods}

    Define string representation via \mono{__repr__()}:

    \begin{cppcode}
        py::class_<Response>(m, "Response")
            ...
            @\hlcode{.def}@("__repr__", [](const Response& self) {
                return std::string()
                    + "<" + std::to_string(self.status)
                    + ": " + self.reason + ">";
            });
    \end{cppcode}

\end{frame}

%%%%%%%%%%%%%%%%%%%%%%%%%%%%%%%%%%%%%%%%%%%%%%%%%%%%%%%%%%%%%%%%%%%%%%%%%%%%%

\begin{frame}[fragile]{Full binding code}

    \begin{cppcode}
        #include <pybind11/operators.h>

        PYBIND11_MODULE(response, m) {
            py::class_<Response>(m, "Response")
                .def(py::init<>())
                .def(py::init<int, std::string>())
                .def(py::init<int, std::string, std::string>())
                .def_readonly("status", &Response::status)
                .def_readonly("reason", &Response::reason)
                .def_readonly("text", &Response::text)
                .def_property_readonly("ok", &Response::ok)
                .def("__repr__", [](const Response& self) {
                    return std::string()
                        + "<" + std::to_string(self.status)
                        + ": " + self.reason + ">";
                })
                .def(py::self == py::self);
        }
    \end{cppcode}

\end{frame}

%%%%%%%%%%%%%%%%%%%%%%%%%%%%%%%%%%%%%%%%%%%%%%%%%%%%%%%%%%%%%%%%%%%%%%%%%%%%%

\begin{frame}[fragile]{Trying it out}

    \begin{pythoncode}
        >>> from response import Response

        >>> Response()
        <200: OK>

        >>> Response().ok
        True

        >>> r = Response(404, 'Not Found')

        >>> r.reason
        'Not Found'

        >>> r.ok
        False

        >>> Response(200, 'OK') == Response()
        True
    \end{pythoncode}

\end{frame}

%%%%%%%%%%%%%%%%%%%%%%%%%%%%%%%%%%%%%%%%%%%%%%%%%%%%%%%%%%%%%%%%%%%%%%%%%%%%%
%%%%%%%%%%%%%%%%%%%%%%%%%%%%%%%%%%%%%%%%%%%%%%%%%%%%%%%%%%%%%%%%%%%%%%%%%%%%%
%%%%%%%%%%%%%%%%%%%%%%%%%%%%%%%%%%%%%%%%%%%%%%%%%%%%%%%%%%%%%%%%%%%%%%%%%%%%%

\section{Function signatures}

%%%%%%%%%%%%%%%%%%%%%%%%%%%%%%%%%%%%%%%%%%%%%%%%%%%%%%%%%%%%%%%%%%%%%%%%%%%%%

\begin{frame}[fragile]{Docstrings and argument names}

    Docstrings can be set by passing string literals to \mono{def()}.\\
    Arguments can be named via \mono{py::arg("...")}.

    \begin{cppcode}
        m.def("greet", [](const std::string& name) {
                py::print("Hello, " + name + ".");
            },
            "Greet a person.",
            py::arg("name")
        );
    \end{cppcode}

    \pause

    \begin{pythoncode}
        >>> greet('stranger')
        Hello, stranger.

        >>> greet?
        greet(name: str) -> None

        Greet a person.
    \end{pythoncode}
\end{frame}

%%%%%%%%%%%%%%%%%%%%%%%%%%%%%%%%%%%%%%%%%%%%%%%%%%%%%%%%%%%%%%%%%%%%%%%%%%%%%

\begin{frame}[fragile]{Keyword arguments with default values}

    Default argument values can be set by assigning to \mono{py::arg()}.

    \begin{cppcode}
        m.def("greet", [](const std::string& name, int times) {
                for (int i = 0; i < times; ++i)
                    py::print("Hello, " + name + ".");
            },
            "Greet a person.",
            py::arg("name"), py::arg("times") = 1
        );
    \end{cppcode}

    \pause

    \begin{pythoncode}
        >>> greet('Jeeves')
        Hello, Jeeves.

        >>> greet('Wooster', times=2)
        Hello, Wooster.
        Hello, Wooster.
    \end{pythoncode}
\end{frame}

%%%%%%%%%%%%%%%%%%%%%%%%%%%%%%%%%%%%%%%%%%%%%%%%%%%%%%%%%%%%%%%%%%%%%%%%%%%%%

\begin{frame}[fragile]{Python objects as arguments}

    Functions can take arbitrary Python objects as arguments:

    \begin{cppcode}
        m.def("count_strings", [](py::list list) {
            int n = 0;
            for (auto item : list)
                if (py::isinstance<py::str>(item))
                    n++;
            return n;
        });
    \end{cppcode}

    \pause

    \begin{pythoncode}
        >>> count_strings(['foo', 'bar', 1, {}, 'baz'])
        3
    \end{pythoncode}
\end{frame}

%%%%%%%%%%%%%%%%%%%%%%%%%%%%%%%%%%%%%%%%%%%%%%%%%%%%%%%%%%%%%%%%%%%%%%%%%%%%%

\begin{frame}[fragile]{*args and **kwargs}

    Variadic positional and keyword arguments can be passed via
    \mono{py::args} (subclass of \mono{py::tuple}) and \mono{py::kwargs}
    (subclass of \mono{py::dict}):

    \begin{cppcode}
        m.def("count_args", [](py::args a, py::kwargs kw) {
            py::print(a.size(), "args,", kw.size(), "kwargs");
        });
    \end{cppcode}

    \pause

    \begin{pythoncode}
        >>> count_args(10, 20, 30, x='a', y='b')
        3 args, 2 kwargs
    \end{pythoncode}
\end{frame}

%%%%%%%%%%%%%%%%%%%%%%%%%%%%%%%%%%%%%%%%%%%%%%%%%%%%%%%%%%%%%%%%%%%%%%%%%%%%%

\begin{frame}[fragile]{Function overloads}

    It is possible to bind multiple \cpp{} overloads to a single
    Python name:

    \begin{cppcode}
        m.def("f", [](int x) { return "int"; });
        m.def("f", [](float x) { return "float"; });
    \end{cppcode}

    \pause

    \begin{pythoncode}
        >>> f(42)
        'int'

        >>> f(3.14)
        'float'

        >>> f('cat')
        TypeError: f(): incompatible function arguments.
        The following argument types are supported:
            1. (arg0: int) -> str
            2. (arg0: float) -> str
    \end{pythoncode}
\end{frame}

%%%%%%%%%%%%%%%%%%%%%%%%%%%%%%%%%%%%%%%%%%%%%%%%%%%%%%%%%%%%%%%%%%%%%%%%%%%%%
%%%%%%%%%%%%%%%%%%%%%%%%%%%%%%%%%%%%%%%%%%%%%%%%%%%%%%%%%%%%%%%%%%%%%%%%%%%%%
%%%%%%%%%%%%%%%%%%%%%%%%%%%%%%%%%%%%%%%%%%%%%%%%%%%%%%%%%%%%%%%%%%%%%%%%%%%%%

\section{Everything else}

%%%%%%%%%%%%%%%%%%%%%%%%%%%%%%%%%%%%%%%%%%%%%%%%%%%%%%%%%%%%%%%%%%%%%%%%%%%%%

\begin{frame}[fragile]{Type conversions}

    Three ways to communicate objects between \cpp{} and Python:

    \begin{enumerate}
        \item \pause
            \textcolor{dgreen}{native} in \cpp{},
            \textcolor{dred}{wrapper} in Python:

            \smallskip
            \begin{cppcode}
                py::class_<Foo>(m, "Foo");
                m.def("f1", [](const Foo& foo) { ... });
            \end{cppcode}

        \item \pause
            \textcolor{dred}{wrapper} in \cpp{},
            \textcolor{dgreen}{native} in Python:

            \smallskip
            \begin{cppcode}
                m.def("f2", [](py::list list) { ... });
            \end{cppcode}

        \item \pause
            \textcolor{dgreen}{native} in \cpp{},
            \textcolor{dgreen}{native} in Python
            (\textit{type conversion}):

            \smallskip
            \begin{cppcode}
                m.def("f3", [](int x) { ... });
                m.def("f4", [](const std::string& s) { ... });
                m.def("f5", [](const std::vector<int>& v) { ... });
            \end{cppcode}

            (always requires a \textit{copy})
    \end{enumerate}

\end{frame}

%%%%%%%%%%%%%%%%%%%%%%%%%%%%%%%%%%%%%%%%%%%%%%%%%%%%%%%%%%%%%%%%%%%%%%%%%%%%%

\begin{frame}{Built-in conversions}

    Some of the supported \cpp{} types:

    \begin{itemize}
        \item Scalar: integer types, \mono{float}, \mono{double}, \mono{bool}, \mono{char}
        \item Strings: \mono{std::string}, \mono{const char *}
        \item Tuples: \mono{std::pair<F, S>}, \mono{std::tuple<...>}
        \item Sequences: \mono{std::vector<T>}, \mono{std::list<T>}, \mono{std::array<T, n>}
        \item Maps: \mono{std::map<K, V>}, \mono{std::unordered_map<K, V>}
        \item Sets: \mono{std::set<T>}, \mono{std::unordered_set<T>}
        \item Polymorphic functions: \mono{std::function<...>}
        \item Date/time: \mono{std::chrono::duration}, \mono{std::chrono::time_point}
        \item Optional: \mono{std::optional<T>}, \mono{std::experimental::optional<T>}
    \end{itemize}

\end{frame}

%%%%%%%%%%%%%%%%%%%%%%%%%%%%%%%%%%%%%%%%%%%%%%%%%%%%%%%%%%%%%%%%%%%%%%%%%%%%%

\begin{frame}[fragile]{Classes}
    \begin{itemize}
        \item \pause Single and multiple inheritance.
        \item \pause Overriding \cpp{} virtual methods from Python.
        \item \pause Custom constructors:
            \smallskip
            \begin{cppcode}
                py::class<A>(m, "A")
                    .def("__init__", [](A& self, int arg) {
                        new (&self) A(arg);  // py::init<int>()
                    });
            \end{cppcode}
        \item \pause Implicit conversions:
            \smallskip
            \begin{cppcode}
                py::class_<A>(m, "A");
                py::class_<B>(m, "B")
                    .def(py::init<A>());
                py::implicitly_convertible<A, B>();
                m.def("f", [](const B& arg) { ... });
            \end{cppcode}
        \item \pause Operator overloading, \mono{py::self} helper.
        \item \pause Static methods, properties, attributes.
    \end{itemize}
\end{frame}

%%%%%%%%%%%%%%%%%%%%%%%%%%%%%%%%%%%%%%%%%%%%%%%%%%%%%%%%%%%%%%%%%%%%%%%%%%%%%

\begin{frame}[fragile]{Python interface}
    \begin{itemize}
        \item \pause Objects with / without refcounting
            (\mono{py::object} / \mono{py::handle})
        \item \pause Built-in types (\mono{py::int_}, \mono{py::list},
            \mono{py::module}, \mono{py::function}, \ldots)
        \item \pause Casting: \mono{py::cast(cpp_obj)}, \mono{py_obj.cast<T>()}.
        \item \pause Calling Python functions via \mono{()},
            \mono{*args} and \mono{**kwargs} unpacking:

            \smallskip
            \begin{cppcode}
                using namespace pybind11::literals;
                auto ship = py::make_tuple("USS Enterprise", 1701);
                auto bridge = py::dict("Jim"_a=1, "Spock"_a=2);
                auto others = py::dict("Scotty"_=4);
                py::function engage = ...;
                engage(*ship, **bridge, "McCoy"_a=3, **others);
            \end{cppcode}

        \item \pause Import modules -- \mono{py::module::import()}.
        \item \pause \mono{print()} function -- \mono{py::print()}.
        \item \pause \mono{str.format()} method -- \mono{py::str::format()}.
        \item \pause \mono{py::len()}, \mono{py::isinstance<>()}, etc.
        \item \pause Run Python code -- \mono{py::eval()}, \mono{py::eval_file()}.
    \end{itemize}
\end{frame}

%%%%%%%%%%%%%%%%%%%%%%%%%%%%%%%%%%%%%%%%%%%%%%%%%%%%%%%%%%%%%%%%%%%%%%%%%%%%%

\begin{frame}[fragile]{Buffer protocol and NumPy}
    \begin{itemize}
        \item \pause Buffer protocol for a type: \mono{.def_buffer()}.
        \item \pause \mono{py::buffer}, \mono{py::memoryview}.
        \item \pause NumPy: \mono{py::array}, \mono{py::array_t<T>}.
        \item \pause Checked (default) or unchecked element access.
        \item \pause Fast access to array properties via NumPy C API.
        \item \pause Support for registering structured NumPy dtypes.
        \item \pause Automatic function vectorization (\mono{py::vectorize}).
        \item \pause Also: Eigen support.
    \end{itemize}
\end{frame}

%%%%%%%%%%%%%%%%%%%%%%%%%%%%%%%%%%%%%%%%%%%%%%%%%%%%%%%%%%%%%%%%%%%%%%%%%%%%%

\begin{frame}[fragile]{... and a few other things}
    \begin{itemize}
        \item \pause Return value policies (\mono{copy}, \mono{move},
            \mono{reference}, \mono{reference_internal},
            \mono{automatic}, \mono{automatic_reference}).
        \item \pause Call policies: \mono{py::keep_alive<Nurse, Patient>}.
        \item \pause Automatic translation of built-in exceptions.
        \item \pause Custom exception translators.
        \item \pause Smart pointers and custom holder types.
        \item \pause \pybind runtime: capsule, registered types map,
            registered instances map.
    \end{itemize}
\end{frame}

%%%%%%%%%%%%%%%%%%%%%%%%%%%%%%%%%%%%%%%%%%%%%%%%%%%%%%%%%%%%%%%%%%%%%%%%%%%%%
%%%%%%%%%%%%%%%%%%%%%%%%%%%%%%%%%%%%%%%%%%%%%%%%%%%%%%%%%%%%%%%%%%%%%%%%%%%%%
%%%%%%%%%%%%%%%%%%%%%%%%%%%%%%%%%%%%%%%%%%%%%%%%%%%%%%%%%%%%%%%%%%%%%%%%%%%%%

\section{Functions and callbacks}

%%%%%%%%%%%%%%%%%%%%%%%%%%%%%%%%%%%%%%%%%%%%%%%%%%%%%%%%%%%%%%%%%%%%%%%%%%%%%

\begin{frame}[fragile]{Functions and callbacks}

    Type conversions for \mono{std::function<...>} can be enabled by
    including an optional \mono{pybind11/functional.h} header.
    Python to \cpp{} callback:

    \begin{cppcode}
        #include <pybind11/functional.h>

        m.def("for_even", [](int n, std::function<void(int)> f) {
            for (int i = 0; i < n; ++i)
                if (i % 2 == 0)
                    f(i);
        });
    \end{cppcode}

    \pause

    \begin{pythoncode}
        >>> def callback(x):
        ...     print('received:', x)

        >>> for_even(3, callback)
        received: 0
        received: 2
    \end{pythoncode}

\end{frame}

%%%%%%%%%%%%%%%%%%%%%%%%%%%%%%%%%%%%%%%%%%%%%%%%%%%%%%%%%%%%%%%%%%%%%%%%%%%%%

\begin{frame}[fragile]{Higher order functions...}

    \begin{cppcode}
        using int_fn = std::function<int(int)>;

        int_fn apply_n(int_fn f, int n) {             // f(f(..(x))
            return [f, n](int x) {
                for (int i = 0; i < n; ++i)
                    x = f(x);
                return x;
            };
        }

        m.def("apply_n", apply_n);
    \end{cppcode}

    \pause

    \begin{pythoncode}
        >>> def f(x):
        ...     return x * 2

        >>> g = apply_n(f, 8)
        >>> g(10)
        2560
    \end{pythoncode}

\end{frame}

%%%%%%%%%%%%%%%%%%%%%%%%%%%%%%%%%%%%%%%%%%%%%%%%%%%%%%%%%%%%%%%%%%%%%%%%%%%%%

\begin{frame}[fragile]{... and higher...}

    \begin{cppcode}
        using int_fn = std::function<int(int)>;

        int_fn @\hlcode{apply_n}@(int_fn f, int n) {             // f(f(..(x))
            return [f, n](int x) {
                for (int i = 0; i < n; ++i)
                    x = f(x);
                return x;
            };
        }

        std::function<int_fn(int_fn)> @\hlcodealt{apply_n}@(int n) { // decorator
            return [n](int_fn f) {
                return @\hlcode{apply_n}@(f, n);
            }
        }

        m.def("apply_n", py::overload_cast<int_fn, int>(@\hlcode{apply_n}@));
        m.def("apply_n", py::overload_cast<int>(@\hlcodealt{apply_n}@));
    \end{cppcode}

\end{frame}

%%%%%%%%%%%%%%%%%%%%%%%%%%%%%%%%%%%%%%%%%%%%%%%%%%%%%%%%%%%%%%%%%%%%%%%%%%%%%

\begin{frame}[fragile]{... decorators?}

    \begin{pythoncode}
        def f(x):
            return x * 2

        >>> apply_n(f, 8)(10)
        2560
    \end{pythoncode}

    \pause

    \begin{pythoncode}
        @apply_n(8)
        def g(x):
            return x * 2

        >>> g(10)
        2560
    \end{pythoncode}

\end{frame}

%%%%%%%%%%%%%%%%%%%%%%%%%%%%%%%%%%%%%%%%%%%%%%%%%%%%%%%%%%%%%%%%%%%%%%%%%%%%%
%%%%%%%%%%%%%%%%%%%%%%%%%%%%%%%%%%%%%%%%%%%%%%%%%%%%%%%%%%%%%%%%%%%%%%%%%%%%%
%%%%%%%%%%%%%%%%%%%%%%%%%%%%%%%%%%%%%%%%%%%%%%%%%%%%%%%%%%%%%%%%%%%%%%%%%%%%%

\section{NumPy example}

%%%%%%%%%%%%%%%%%%%%%%%%%%%%%%%%%%%%%%%%%%%%%%%%%%%%%%%%%%%%%%%%%%%%%%%%%%%%%

\begin{frame}[fragile]{NumPy example: rolling stats}

\begin{minted}[fontsize=\smaller]{cpp}
#include <pybind11/numpy.h>

struct Stats { double mean; double std; };

auto rolling_stats(py::array_t<double> arr, size_t window) {
    if (arr.ndim() != 1)
        throw std::runtime_error("expected 1-D array");

    py::array_t<Stats> stats(arr.size());
    auto a = arr.unchecked<1>();
    auto s = stats.mutable_unchecked<1>();

    double sum = 0, sqr = 0;
    for (size_t i = 0; i < arr.size(); ++i) {
        if (i >= window) {
            auto x = a(i - window); sum -= x; sqr -= x * x;
        }
        auto x = a(i); sum += x; sqr += x * x;
        double n = i >= window ? window : (i + 1)
        double mean = sum / n;
        s(i) = { mean, std::sqrt((sqr - sum * mean) / (n - 1)) };
    }
    return stats;
}

PYBIND11_MODULE(example, m) {
    PYBIND11_NUMPY_DTYPE(Stats, mean, std);
    m.def("rolling_stats", rolling_stats);
}
\end{minted}

\end{frame}

%%%%%%%%%%%%%%%%%%%%%%%%%%%%%%%%%%%%%%%%%%%%%%%%%%%%%%%%%%%%%%%%%%%%%%%%%%%%%

\begin{frame}[fragile]{NumPy example: rolling stats}

    \begin{pythoncode}
        >>> import pandas as pd
        >>> pd.DataFrame(rolling_stats([1, 4 9, 16, 25], window=2))
    \end{pythoncode}

    \pause

    \begin{center}
        \vspace{-0.2cm}\ttfamily\scriptsize
        \begin{tabular}{lrr} \toprule
                  & mean & std      \\ \midrule
                0 &  1.0 &  NaN     \\
                1 &  2.5 & 2.121320 \\
                2 &  6.5 & 3.535534 \\
                3 & 12.5 & 4.949747 \\
                4 & 20.5 & 6.363961 \\ \bottomrule
        \end{tabular}
        \vspace{-0.2cm}
    \end{center}

    \pause

    Correctness check:

    \begin{pythoncode}
        import numpy as np

        a = np.random.random(25 * 1000 * 1000)
        stats = rolling_stats(a, window=1000)
        rolling = pd.Series(a).rolling(window=1000, min_periods=0)

        assert np.allclose(stats['mean'], rolling.mean())
        assert np.allclose(stats['std'], rolling.std(), equal_nan=1)
    \end{pythoncode}

\end{frame}

%%%%%%%%%%%%%%%%%%%%%%%%%%%%%%%%%%%%%%%%%%%%%%%%%%%%%%%%%%%%%%%%%%%%%%%%%%%%%

\begin{frame}[fragile]{NumPy example: rolling stats}
    Performance check:

    \begin{pythoncode}
        a = np.random.random(25 * 1000 * 1000)
        stats = rolling_stats(a, window=1000)
        rolling = pd.Series(a).rolling(window=1000, min_periods=0)
    \end{pythoncode}

    \pause \darrow

    \begin{pythoncode}
        >>> %timeit rolling.mean()
        1.1 s @$\pm$@ 24.9 ms per loop
    \end{pythoncode}

    \pause

    \begin{pythoncode}
        >>> %timeit rolling.std()
        1.18 s @$\pm$@ 16.3 ms per loop
    \end{pythoncode}

    \pause

    \begin{pythoncode}
        >>> %timeit rolling_stats(a, 1000)
        264 ms @$\pm$@ 4.36 ms per loop
    \end{pythoncode}
\end{frame}

%%%%%%%%%%%%%%%%%%%%%%%%%%%%%%%%%%%%%%%%%%%%%%%%%%%%%%%%%%%%%%%%%%%%%%%%%%%%%
%%%%%%%%%%%%%%%%%%%%%%%%%%%%%%%%%%%%%%%%%%%%%%%%%%%%%%%%%%%%%%%%%%%%%%%%%%%%%
%%%%%%%%%%%%%%%%%%%%%%%%%%%%%%%%%%%%%%%%%%%%%%%%%%%%%%%%%%%%%%%%%%%%%%%%%%%%%

\section{Thanks}

%%%%%%%%%%%%%%%%%%%%%%%%%%%%%%%%%%%%%%%%%%%%%%%%%%%%%%%%%%%%%%%%%%%%%%%%%%%%%

\begin{frame}{Thanks}
    Wenzel Jakob (\mono{@wjakob}) -- for creating this awesome project.

    \pause

    Jason Rhinelander (\mono{@jagerman}) and Dean Moldovan
    (\mono{@dean0x7d}) -- for maintaining it and adding a metric
    ton of features and tests.

    \pause

    Jonas Adler, Sylvain Corlay, Trent Houliston, Axel Huebl, \mono{@hulucc},
    Sergey Lyskov, Johan Mabille, Tomasz Mi\k{a}sko, Ben Pritchard,
    Boris Sch\"{a}ling, Pim Schellart, Patrick Stewart and myself
    (\mono{@aldanor}) -- for contributing significant features or improvements.

    \pause

    Dave Abrahams -- for creating Boost.Python.

    \hrulefill

    \pause

    \begin{center}
        Thanks for listening!
    \end{center}
\end{frame}

\end{document}
